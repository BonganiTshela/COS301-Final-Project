\documentclass[runningheads,a4paper]{article}

\usepackage[utf8]{inputenc}

\setcounter{tocdepth}{3}

\usepackage[english]{babel} 
\usepackage{graphicx}
\usepackage{grffile}
\usepackage{float}
\usepackage{multicol}
\usepackage{url}
\usepackage{array}
\usepackage{wrapfig}
 \usepackage{multirow}
\usepackage{tabu}
\usepackage{amssymb}% http://ctan.org/pkg/amssymb
\usepackage{pifont}% http://ctan.org/pkg/pifont
\usepackage[font=small,labelfont=bf]{caption}

\newcommand{\cmark}{\ding{51}}%
\newcommand{\xmark}{\ding{55}}%

\usepackage{titling}
\usepackage[hidelinks]{hyperref}
\setcounter{secnumdepth}{5}
%Margins
\usepackage[
margin=2cm,
includefoot
]{geometry}


\graphicspath{{/}}

%Headers and Footers
\usepackage{fancyhdr}
\pagestyle{fancy}
\fancyhead{}
\fancyfoot{}
\fancyfoot[R]{\thepage}
\renewcommand{\headrulewidth}{0pt}
\renewcommand{\footrulewidth}{0pt}
\setlength\parindent{24pt}
\begin{document}
	
%Title Page
\begin{titlepage}
	\begin{center}
		\includegraphics[width=10cm]{Images\UP_Logo.PNG}  \\
		[1cm]
		\line(1,0){300} \\
		[0.3cm]
		\textsc{\Large
			Benchmarking Service\\
			User Manual\\
			\hfill
			%University of Pretoria
		}\\
		[0.1cm]
		\line(1,0){300} \\
		[0.7cm]
		\textsc{\Large
			ProCoders
		} \\
	\end{center}
	
	\begin{center}
		\begin{centre}
			\textsc{\large\\
				Bongani Tshela - 14134790\\ 
			}
		
			\textsc{\large\\
				Harris Leshaba - 15312144\\ 
			}

			\textsc{\large\\
				Joseph Letsoalo - 15043844\\ 
			}
			
			\textsc{\large\\
				Minal Pramlall - 13288157\\ 
			}
		

            

		\end{centre}
		
		
		
	\end{center}
\end{titlepage}
%\maketitle

\begingroup

\tableofcontents
\addcontentsline{toc}{section}{Table Of Contents}
\endgroup
\newpage

\section{Introduction}
Benchmarking is a very common and useful service, and yet few benchmarking tools or services are easily available. Those that are available may require intricate configuration that is beyond the reach of the budding developers, researchers, teachers and students who would like to use them. The development of a benchmarking service which can be used in a simple and generic way would therefore be welcomed by a large potential user base.


\section{Getting Started}
	\subsection{Quick Start}
	Open up the benchmarking website on a web browser, register/log in to your account or simply just proceed to benchmarking. Upload your algorithms and submit for benchmarking, you will automatically be shown your results and may choose to save it if you are registered.
	
	\subsection{Main Scenarios of Use}
	Whether you are a student or even a researcher that would want to use this system to find the optimal algorithm for reaching a solution, this system is easy to use and available to any type of user.
	
	\subsection{System Requirements}
	Internet Browser (Chrome/Firefox/Opera) and a working internet connection.
	
\section{Activity1}

\section{Activity2}

%----------------- Continue for all activities that can be performed

\end{document}
