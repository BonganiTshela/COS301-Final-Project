\documentclass[runningheads,a4paper]{article}

\usepackage[utf8]{inputenc}

\setcounter{tocdepth}{3}

\usepackage[english]{babel} 
\usepackage{graphicx}
\usepackage{grffile}
\usepackage{float}
\usepackage{multicol}
\usepackage{url}
\usepackage{array}
\usepackage{wrapfig}
 \usepackage{multirow}
\usepackage{tabu}
\usepackage{amssymb}% http://ctan.org/pkg/amssymb
\usepackage{pifont}% http://ctan.org/pkg/pifont
\usepackage[font=small,labelfont=bf]{caption}

\newcommand{\cmark}{\ding{51}}%
\newcommand{\xmark}{\ding{55}}%

\usepackage{titling}
\usepackage[hidelinks]{hyperref}
\setcounter{secnumdepth}{5}
%Margins
\usepackage[
margin=2cm,
includefoot
]{geometry}


\graphicspath{{/}}

%Headers and Footers
\usepackage{fancyhdr}
\pagestyle{fancy}
\fancyhead{}
\fancyfoot{}
\fancyfoot[R]{\thepage}
\renewcommand{\headrulewidth}{0pt}
\renewcommand{\footrulewidth}{0pt}
\setlength\parindent{24pt}
\begin{document}
	
	%Title Page
	\begin{titlepage}
		\begin{center}
			\includegraphics[width=10cm]{UP_Logo.PNG}  \\
			[1cm]
			\line(1,0){300} \\
			[0.3cm]
			\textsc{\Large
				Object Sensor and Mobile Navigation\\
				Tender Document\\
				\hfill
				%University of Pretoria
			}\\
			[0.1cm]
			\line(1,0){300} \\
			[0.7cm]
			\textsc{\Large
				ProCoders
			} \\
		\end{center}
		
		\begin{center}
			\begin{centre}
				\textsc{\large\\
					Bongani Tshela - 14134790\\ 
				}
			
    				\textsc{\large\\
    					Harris Leshaba - 15312144\\ 
    				}
				\textsc{\large\\
					Joseph Letsoalo - 15043844\\ 
				}
				
				\textsc{\large\\
					Minal Pramlall - 13288157\\ 
				}
				\textsc{\large\\
					Mandla Mhlongo - 29630135\\ 
				}
			\end{centre}
			
			
			%\textsc{\\
			%	All Testing can be found here: \\ \href{https://github.com/SirJosh/Android-GIS/tree/master/Testing-Phase4}{GitHub}
			%	\url{https://github.com/SirJosh/Android-GIS/tree/master/Testing-Phase4}}
		\end{center}
	\end{titlepage}
	%\maketitle
	
	\begingroup
	
	\tableofcontents
	\addcontentsline{toc}{section}{Table Of Contents}
	\endgroup
	\newpage
	
	\section{Description} %-------The description starts here-----------------------%
	
	\subsection{Technologies}
	We intend to address the requirements of the application by using open source resources as much as we can.
Students use the same facilities most of the time, often in a certain order. This is due to a well-defined schedule (timetable) that students follow. Thus the ability of the application to collect and store these patterns will make the application very efficient and effective to use. We intend to use PostgreSQL to store the relevant behavioural data needed. Since most mobile use iOS or android operating systems, the application will run on both platforms. A Web front will also be another access method. We intend to use the application shell architecture in order to develop for all access method concurrently.\\
Since the application aims to cater for students with visual disabilities, we will use google assistant speech API to provide voice commands to help these student to navigate the campus safely. Bluetooth tags, that enables the localisation and distance of an obstacle, will be used together with Bluetooth transmitters to communicate with a student’s mobile device to send information about the student’s position and determine distance from obstacle. Google proximity beacon API will be used in this regard to give our users the location and proximity of a Bluetooth tag, which represent an obstacle.\\ 

  \section{Methodology} %-------The Methodology starts here-----------------------%
  \section{Team members details}%-----More details about the team-----------------%
  \end{document}
